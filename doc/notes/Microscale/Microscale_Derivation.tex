\chapter{Microscale Derivation}
\chapterauthor{V. W. L. Chan}
\section{Motivation}

This chapter deals with the development of the microscale analysis. For the current use case this is limited to a fiber network, where each fiber is made up of a truss element. Due to the limited use case, this chapter will also discuss the fiber constitutive properties. \todo{do we want to move fiber constitutive properties to the constitutive property section?}

It describes volume averaged stress of the network (which maps to the Cauchy stress at the macroscale) and derivative of the volume averaged stress with respect to the macroscale finite element (FE) nodes. These quantities are calculated in the microscale portion of the code. The stress and stress derivative values are used in the Newton-Raphson method in the macroscale portion of the code.


\section{Calculation of volume-averaged stress}
\label{sec:macrostress}

According to Stylianopoulos and Barocas, the macroscopic stress tensor, $S_{ij}$, is determined from the forces, $T_j^f$, that act on the nodes $f$ of a fiber element in the direction $j$ through a volume averaging process given by
%
\begin{equation}
S_{ij} = \frac{1}{V} \sum_{\text{bcl}} x_i^f T_j^f,
\label{S_ij}
\end{equation}
%
where the summation is over fiber nodes that lie on the boundary faces of a representative volume element (RVE) and $bcl$ stands for boundary cross link. The volume average is over the current volume of the RVE, $V$, and $x_i^f$ is the $i^{th}$-component (i=1,2, or 3) of the positions of fiber node $f$ that is on the boundary face of the RVE \cite{Stylianopoulos:2007dp,nemat-nasser_micromechanics:1999}.

\subsection{Microscale forces on fiber nodes}

Prior to the volume averaging described in Eq.\ \eqref{S_ij}, the forces that act on the fiber nodes in the RVE must be calculated and are determined from the equilibrium nodal positions of each fiber in the RVE via 
%
\begin{equation}
F \equiv \frac{E A}{B}[\exp(B \varepsilon) - 1],
\label{fiber_force}
\end{equation}
%
where $E$ is the linear modulus of the fiber, $A$ is the cross-sectional area of the fiber, $B$ is a constant, and $\varepsilon$ is the fiber's Green strain (along its axis).

Rotating the green strain along the principal axis of the truss, we see that the green strain can be written as:
%
\begin{equation}
\varepsilon \equiv \frac{1}{2} \left(\lambda^2-1\right),
\label{Green_strain}
\end{equation}
% 
where $\lambda$ is the fiber stretch ratio \cite{Chandran:2007hy}. \todo{Should we bother with a discussion on how the force is not the "real force" here? e.g. GL strain gives PK2 Stress}

The force on fiber node $b$, $\textbf{T}^b$, is defined in terms of the scalar $F$ as
%
\begin{equation}
\textbf{T}^b = F \textbf{n},
\label{fiber_node_force}
\end{equation}
%
where $\textbf{T}^b = T^b_i \textbf{e}_i$ and the unit vector $\textbf{n}$ is determined from the current end coordinates, $\textbf{x}^a$ and $\textbf{x}^b$ at nodes $a$ and $b$ of a fiber, respectively, by
%
\begin{equation}
\textbf{n} = \frac{1}{l(\textbf{x})}(\textbf{x}^b - \textbf{x}^a),
\end{equation}
%
where $\textbf{x} = x_i \textbf{e}_i$, the superscript indicates the node of which the position describes, and $l(\textbf{x})$ is the current length of the fiber given by
%
\begin{equation}
l(\textbf{x}) = \sqrt{(\textbf{x}^b - \textbf{x}^a) \cdot (\textbf{x}^b - \textbf{x}^a)}.
\label{fiber_length}
\end{equation}
%
Note that the definitions in Eqs.\ \eqref{fiber_node_force} through \eqref{fiber_length} are based on the convention that the force acting on nodes $b$ and $a$ are positive and negative, respectively, when the corresponding fiber element is loaded in tension.

Consequently, $\textbf{T}^a = - \textbf{T}^b$. The equilibrium fiber node positions in a RVE ($\textbf{x}^b$ and $\textbf{x}^a$ of each fiber) are determined via a Newton-Raphson method.

\subsection{Microscale Tangent Stiffness Tensor}
\label{subsec:tangent_stiffness_tensor}

\todo{Need to add the linear constitutive as well}
In order to find the equilibrium fiber-node positions in the RVEs via the Newton-Raphson procedure, we need to define the tangent stiffness tensor. The tangent stiffness tensor is determined by taking the directional derivative of $\textbf{T}^b$ (we can equivalently take the directional derivative of $\textbf{T}^a$, in which case the result will be equal and opposite) in the direction of the displacement of the fiber, $\textbf{u} \equiv \textbf{u}^b - \textbf{u}^a$:
%
\begin{eqnarray}
D\textbf{T}^b [\textbf{u}] &=& D F \textbf{n} [\textbf{u}] \nonumber\\
&=& \frac{E A}{B} \bigg(D \{ \exp(B \varepsilon) - 1\} [\vec{u}] \vec{n}+ D \vec{n} [\vec{u}] \{ \exp(B \varepsilon) - 1\} \bigg) \nonumber\\
&=& \frac{E A}{B} \bigg( R \vec{n} + \vec{S}\{\exp(B \mat{\varepsilon}) - 1\} \bigg),
\label{DT_i}
\end{eqnarray}
%
where the notation $D f(\vec{x}) [\vec{u}]$ denotes the directional derivative of $f(\vec{x})$ in the direction of $\vec{u}$, and 
% 
\begin{equation}
R \equiv D \{ \exp(B \varepsilon) - 1\} [\vec{u}] \ \ \text{and} \ \
\vec{S} \equiv D \vec{n} [\vec{u}].
\end{equation}
%

To obtain a more useful form of Eq.\ \eqref{DT_i}, $\vec{R}$ and $\mat{S}$ can be written out explicitly by expanding the directional derivative and using $\lambda \equiv l(\vec{x})/L$, and Eq. \eqref{Green_strain}: \todo{Write in general terms first, then give the constitutive properties. Similar to write up I did on truss integrator.}
%
\begin{eqnarray}
D \{ \exp(B \varepsilon) - 1\} [\vec{u}] &=& B \exp(B \varepsilon) D \varepsilon [\vec{u}] \nonumber\\
&=& \frac{0.5B}{L^2} \exp(B \varepsilon) D( l^2(\vec{x}) - 1) [\vec{u}]\nonumber\\
&=& \frac{B}{L^2}\exp(B \varepsilon)   l(\vec{x})\vec{n} \cdot (\vec{u}^b - \vec{u}^a) \nonumber\\
&=& R
\label{R}
\end{eqnarray}
%
We also have:
%
\begin{eqnarray}
D \textbf{n} [\textbf{u}]  &=& D \frac{1}{l(\textbf{x})}(\textbf{x}^b - \textbf{x}^a) [\textbf{u}] \nonumber\\
&=& D \frac{1}{l(\textbf{x})} [\textbf{u}](\textbf{x}^b - \textbf{x}^a) + D (\textbf{x}^b - \textbf{x}^a) [\textbf{u}] \frac{1}{l(\textbf{x})} \nonumber\\
&=& -\frac{1}{l^{2}(\textbf{x})}\textbf{n} \cdot (\textbf{u}^b - \textbf{u}^a)(\textbf{x}^b - \textbf{x}^a) + (\textbf{u}^b - \textbf{u}^a)  \frac{1}{l(\textbf{x})} \nonumber\\
&=& \frac{(\textbf{u}^b - \textbf{u}^a)-\textbf{n} \cdot (\textbf{u}^b - \textbf{u}^a)\textbf{n}}{l(\textbf{x})} \nonumber\\
&=& \vec{S}.
\label{S}
\end{eqnarray}
%
In Eqs.\ \eqref{R} and \eqref{S}, we have used the relationships listed in Appendix \ref{app:relationships}. Finally, substituting Eqs.\ \eqref{R} and \eqref{S} into Eq.\ \eqref{DT_i} we arrive at:
%
\begin{eqnarray}
D\textbf{T}^b [\textbf{u}] &=& \frac{E A}{B} \bigg[ \left(\frac{B }{L^2} \exp(B \varepsilon) l(\textbf{x})\textbf{n} \cdot (\textbf{u}^b - \textbf{u}^a)\right)\textbf{n} \nonumber\\
&+& \left( \frac{(\textbf{u}^b - \textbf{u}^a)-\textbf{n} \cdot (\textbf{u}^b - \textbf{u}^a)\textbf{n}}{l(\textbf{x})} \right) \{ \exp(B \varepsilon) - 1 \} \bigg] \nonumber\\
&=& \bigg[ \frac{E A }{L^2} \exp(B \varepsilon)  l(\textbf{x})-\frac{F}{l(\textbf{x})} \bigg] (\textbf{n} \cdot (\textbf{u}^b - \textbf{u}^a)\textbf{n}) + \frac{F}{l(\textbf{x})}(\textbf{u}^b - \textbf{u}^a) \nonumber\\
&=& \bigg[ \frac{E A  l(\textbf{x})}{L^2} \exp(B \varepsilon) -\frac{F}{l(\textbf{x})} \bigg] (\textbf{n} \otimes \textbf{n})_{3\times3}(\textbf{u}^b - \textbf{u}^a) + \frac{F}{l(\textbf{x})} \text{\textbf{I}}_{3\times3} \cdot (\textbf{u}^b - \textbf{u}^a) , \nonumber\\
\label{DTi_final}
\end{eqnarray}
%
where the property of the tensor product,
%
\begin{equation}
(\textbf{u} \otimes \textbf{v})\textbf{w} = (\textbf{w} \cdot \textbf{v})\textbf{u}
\end{equation}
%
has been used.

\subsubsection{Matrix Equations}
Also note,  the directional derivative for node $a$ is $D \textbf{T}^a(\textbf{x})[\textbf{u}] = - D\textbf{T}^b(\textbf{x})[\textbf{u}]$ since \(\vec{T}^a(\vec{x})=-\vec{T}^b(\vec{x})\).

Equation \eqref{DTi_final} can be rearranged into matrix form. For a single fiber element e, the directional derivative in matrix form is 
%
\begin{eqnarray}
D\textbf{T}^{(\text{e})}(\textbf{x}^{(\text{e})})[\textbf{u}^{(\text{e})}] = \textbf{K}^{(\text{e})}\textbf{u}^{(\text{e})} = 
\begin{bmatrix} 
\textbf{K}^{(\text{e})}_{aa} & \textbf{K}^{(\text{e})}_{ab} \\ 
\textbf{K}^{(\text{e})}_{ba}& \textbf{K}^{(\text{e})}_{bb} \\
\end{bmatrix}
\begin{bmatrix} 
\textbf{u}^{a}  \\ 
\textbf{u}^{b} \\ 
\end{bmatrix},
\label{DT^e_matrix}
\end{eqnarray}
%
where
%
\begin{equation}
\textbf{K}^{(\text{e})}_{aa} = \textbf{K}^{(\text{e})}_{bb} = k (\textbf{n} \otimes \textbf{n})_{3x3} + \frac{F}{l(\textbf{x})}\textbf{\text{I}}_{3x3}, \ \ \text{and} \ \ \textbf{K}^{(\text{e})}_{ab} = \textbf{K}^{(\text{e})}_{ba} = -\textbf{K}^{(\text{e})}_{bb},
\label{K_aa}
\end{equation}
%
and
%
\begin{equation}
k \equiv \frac{E A  l(\textbf{x})}{L^2} \exp(B \varepsilon) -\frac{F}{l(\textbf{x})}.
\label{k}
\end{equation}
%
Additionally, 
%
\begin{eqnarray}
(\textbf{n} \otimes \textbf{n})_{3x3} = 
\begin{bmatrix}
\cos(\alpha)\cos(\alpha) & \cos(\alpha) \cos(\beta) & \cos(\alpha) \cos(\gamma) \\
\cos(\beta)\cos(\alpha) & \cos(\beta)\cos(\beta) & \cos(\beta)\cos(\gamma) \\
\cos(\gamma)\cos(\alpha) & \cos(\gamma)\cos(\beta) & \cos(\gamma)\cos(\gamma) 
\end{bmatrix} \ \ \text{and} \ \
%
\textbf{\text{I}}_{3x3} =
\begin{bmatrix}
1 & 0 & 0 \\
0 & 1 & 0 \\
0 & 0 & 1
\end{bmatrix},
\end{eqnarray}
%
with
%
\begin{equation}
\cos(\alpha) = \frac{x^b_1 - x^a_1}{l(\textbf{x})}, \ \ \cos(\beta) = \frac{x^b_2 - x^a_2}{l(\textbf{x})}, \ \ \text{and} \cos(\gamma) = \frac{x^b_3 - x^a_3}{l(\textbf{x})} 
\end{equation}
%
and \todo{this should be made more clear what is $e_i$}
%
\begin{equation}
\textbf{u}^a = (x^a_i - x^b_i) \textbf{e}_i \ \ \text{and} \ \ \textbf{u}^b = (x^b_i - x^a_i) \textbf{e}_i .
\end{equation}
%
Therefore, $\textbf{u}_a = -\textbf{u}_b$. Using the above relationships, Eq.\ \eqref{K_aa} can be written in matrix form as
%
\begin{eqnarray}
\textbf{K}^{(\text{e})}_{aa} =
\begin{bmatrix}
k \cos(\alpha)\cos(\alpha) + \frac{F}{l(\textbf{x})} & k \cos(\alpha)\cos(\beta) & k \cos(\alpha)\cos(\gamma) \\
k \cos(\beta)\cos(\alpha) & k \cos(\beta)\cos(\beta)+ \frac{F}{l(\textbf{x})}  & k \cos(\beta)\cos(\gamma) \\
k \cos(\gamma)\cos(\alpha) & k \cos(\gamma)\cos(\beta) & k \cos(\gamma)\cos(\gamma) + \frac{F}{l(\textbf{x})} 
\end{bmatrix}
\label{K^e_aa_matrix}
\end{eqnarray}
%
The Newton-Raphson method for a single fiber element is
%
\begin{equation}
\textbf{K}^{(\text{e})} \delta \textbf{u}^{(\text{e})} = - \textbf{T}^{(\text{e})}, \ \text{where} \ \textbf{u}^{(\text{e}),t+1}  = \textbf{u}^{(\text{e}),t} + \delta \textbf{u}^{(\text{e})},
\label{Newton_Raphson}
\end{equation}
%
and the superscripts $t+1$ and $t$ denote the current and next iterations, respectively. In matrix form, $\textbf{u}^{(\text{e})}$ and $\textbf{T}^{(\text{e})}$ are
%
\begin{eqnarray}
\textbf{u}^{(\text{e})} = 
\begin{bmatrix}
\textbf{u}^a \\ \textbf{u}^b
\end{bmatrix}^{(\text{e})} =
%
\begin{bmatrix}
x^a_1 \\ x^a_2 \\ x^a_3 \\ x^b_1 \\ x^b_2 \\ x^b_3
\end{bmatrix}^{(\text{e})} \ \ \text{and} \ \
%%%%%
\textbf{T}^{(\text{e})} = 
\begin{bmatrix}
\textbf{T}^a \\ \textbf{T}^b
\end{bmatrix}^{(\text{e})} =
%
F\begin{bmatrix}
\frac{(x^a_1 - x^b_1)}{l(\textbf{x})} \\ \frac{(x^a_2 - x^b_2)}{l(\textbf{x})} \\ \frac{(x^a_3 - x^b_3)}{l(\textbf{x})} \\ \frac{(x^b_1 - x^a_1)}{l(\textbf{x})} \\ \frac{(x^b_2 - x^a_2)}{l(\textbf{x})} \\ \frac{(x^b_3 - x^a_3)}{l(\textbf{x})}
\end{bmatrix}^{(\text{e})} = 
%
F\begin{bmatrix}
-\cos(\alpha) \\ -\cos(\beta) \\ -\cos(\gamma) \\ \cos(\alpha) \\ \cos(\beta) \\ \cos(\gamma)
\end{bmatrix}^{(\text{e})} ,
\label{u_T_matrix}
\end{eqnarray}
%
where the superscript (e) denotes that Eq.\ \eqref{u_T_matrix} is for a single fiber element.
\todo{show summation of elemental equations}

Once the equilibrium fiber-node positions in a RVE are determined, the forces of the fiber nodes can be volume averaged as in Eq.\ \eqref{S_ij} to calculate the macroscopic stresses.
%
Noting $S_{ij} = S_{ji}$ (e.g. the Cauchy stress tensor is symmetric), there are only six unique terms. Therefore, we can represent the \(\mat{S}\) as a vector.
%
\begin{eqnarray}
\begin{bmatrix}
S_{ij}
\end{bmatrix} = 
%
\begin{bmatrix}
S_{11} \\ S_{12} \\ S_{13} \\ S_{22} \\ S_{23} \\ S_{33} 
\end{bmatrix} =
%
\frac{F}{V} \begin{bmatrix}
\sum_{\text{bcl}} \left[ - x^a_1  \cos(\alpha) + x^b_1 \cos(\alpha) \right] \\ 
%
\frac{1}{2}\sum_{\text{bcl}}  \left[-\left( x^a_1  \cos(\beta)+ x^a_2 \cos(\alpha) \right) + \left( x^b_1  \cos(\beta)+ x^b_2 \cos(\alpha) \right) \right] \\ 
%
\frac{1}{2}\sum_{\text{bcl}} \left[ -\left(x^a_1 \cos(\gamma) + x^a_3 \cos(\alpha) \right) + \left(x^b_1 \cos(\gamma) + x^b_3 \cos(\alpha) \right) \right] \\ 
%
\sum_{\text{bcl}} \left[-x^a_2 \cos(\beta) + x^b_2 \cos(\beta) \right]\\
%
\frac{1}{2} \sum_{\text{bcl}} \left[-\left(x^a_2 \cos(\gamma) + x^a_3 \cos(\beta) \right) + \left(x^a_2 \cos(\gamma) + x^a_3 \cos(\beta) \right)\right] \\
%
\sum_{\text{bcl}} \left[ -x^a_3 \cos(\gamma) + x^b_3 \cos(\gamma) \right]
\end{bmatrix}  ,
\label{S_ij_vector}
\end{eqnarray}
%
\section{Derivative of Macroscopic Stress With Respect to FE Node Positions}

As discussed in Section \ref{sec:macrostress}, the macroscale stresses are calculated from the equilibrium fiber-node positions of each fiber element within a RVE. Since the macroscale stresses explicitly depend on the fiber-node positions, a natural first step is to consider the derivative of the macroscopic stress with respect to the fiber-node positions.

\subsection{Derivative of Macroscopic Stress With Respect to Fiber-Node Positions}

The directional derivative of $S_{ij}$ is
%
\begin{eqnarray}
DS_{ij}[\textbf{u}] &=& D\left( \frac{1}{V} \sum_{\text{bcl}} x_i^f T_j^f \right)[\textbf{u}] \nonumber\\
%
&=& D \left(\frac{1}{V}\right) [\textbf{u}]\sum_{\text{bcl}} x_i^f T_j^f + \frac{1}{V} \sum_{\text{bcl}} \left[ x_i^f D T_j^f[\textbf{u}] + D x_i^f  [\textbf{u}]  T_j^f \right]\nonumber\\
%
&=& D \left(\frac{1}{V}\right) [\textbf{u}] V S_{ij} + \frac{1}{V} \sum_{\text{bcl}}  C_{ij}^f,
\label{DS_ij}
\end{eqnarray}
%
where 
%
\begin{eqnarray}
C_{ij}^f &\equiv&   x_i^f D T_j^f[\textbf{u}] + D x_i^f  [\textbf{u}]  T_j^f \nonumber\\
&=& x_i^f \frac{\partial T_j^f}{\partial x_k^f} u_k^f + \frac{\partial x_i^f}{\partial x_k^f}u_k^f T_j^f ,
\label{C_ij}
\end{eqnarray}
%
where $C_{ij}^f$ is for a single fiber node (as indicated by the superscript $f$). The summation in Eq.\ \eqref{DS_ij} account for all fiber nodes that lie on the RVE boundary. As seen in Eq.\ \eqref{DS_ij}, the directional derivative of $S_{ij}$ requires the calculation of $C_{ij}^f$ and the directional derivative of $1/V$. 

\subsubsection{Calculation of $C_{ij}^f$}

The term $C_{ij}^f$ in Eq.\ \eqref{C_ij} can be explicitly written out as
%
\begin{eqnarray}
\left[ C_{ij}^f \right] &=& 
\begin{bmatrix}
x_1^f \frac{\partial T_1^f}{\partial x_k^f} u_k^f + \frac{\partial x_1^f}{\partial x_k^f}u_k^f T_1^f  & x_1^f \frac{\partial T_2^f}{\partial x_k^f} u_k^f + \frac{\partial x_1^f}{\partial x_k^f}u_k^f T_2^f & x_1^f \frac{\partial T_3^f}{\partial x_k^f} u_k^f + \frac{\partial x_1^f}{\partial x_k^f}u_k^f T_3^f\\
%
x_2^f \frac{\partial T_1^f}{\partial x_k^f} u_k^f  + \frac{\partial x_2^f}{\partial x_k^f} u_k^fT_1^f & x_2^f \frac{\partial T_2^f}{\partial x_k^f} u_k^f  + \frac{\partial x_2^f}{\partial x_k^f} u_k^fT_2^f & x_2^f \frac{\partial T_3^f}{\partial x_k^f} u_k^f  + \frac{\partial x_2^f}{\partial x_k^f} u_k^fT_3^f \\
%
x_3^f \frac{\partial T_1^f}{\partial x_k^f} u_k^f  + \frac{\partial x_3^f}{\partial x_k^f}u_k^f T_1^f & x_3^f \frac{\partial T_2^f}{\partial x_k^f} u_k^f + \frac{\partial x_3^f}{\partial x_k^f} u_k^f T_2^f & x_3^f \frac{\partial T_3^f}{\partial x_k^f} u_k^f + \frac{\partial x_3^f}{\partial x_k^f}u_k^f T_3^f
\end{bmatrix} \nonumber\\
%
&=&
%
\begin{bmatrix}
x_1^f \frac{\partial T_1^f}{\partial x_k^f} u_k^f +  u_1^f T_1^f  & x_1^f \frac{\partial T_2^f}{\partial x_k^f} u_k^f + u_1^f T_2^f & x_1^f \frac{\partial T_3^f}{\partial x_k^f} u_k^f + u_1^f T_3^f\\
%
x_2^f \frac{\partial T_1^f}{\partial x_k^f} u_k^f  +  u_2^f T_1^f & x_2^f \frac{\partial T_2^f}{\partial x_k^f} u_k^f  + u_2^f T_2^f & x_2^f \frac{\partial T_3^f}{\partial x_k^f} u_k^f  + u_2^f T_3^f \\
%
x_3^f \frac{\partial T_1^f}{\partial x_k^f} u_k^f  + u_3^f T_1^f & x_3^f \frac{\partial T_2^f}{\partial x_k^f} u_k^f + u_3^f T_2^f & x_3^f \frac{\partial T_3^f}{\partial x_k^f} u_k^f + u_3^f F_3^f
\label{C_ij_matrix}
\end{bmatrix} ,
%
\end{eqnarray}
%
where 
%
\begin{equation}
T_1^a \equiv -F \cos(\alpha), \ \ T_2^a \equiv -F \cos(\beta), T_3^a \equiv -F \cos(\gamma), \ \ \text{and} \ \ T_i^a = -T_i^b,
\end{equation}
%
and the second equality in Eq.\ \eqref{C_ij_matrix} uses the relationship
%
\begin{equation}
\frac{\partial x_i^f}{\partial x_k^f} = \delta_{ik} .
\end{equation}
%

Since $C_{ij}^f = C_{ji}^f$ in Eq.\ \eqref{C_ij_matrix}, only the six terms on the upper triangle of the matrix are unique. Consequently, the matrix in Eq.\ \eqref{C_ij_matrix} can be rewritten as a matrix-vector product
\todo{write this in standard voigt notation}
%
\begin{eqnarray}
\begin{bmatrix}
C_{11}^f \\ C_{12}^f \\ C_{13}^f \\ C_{22}^f \\
C_{23}^f \\ C_{33}^f 
\end{bmatrix}  = 
%
\begin{bmatrix}
x_1^f \frac{\partial T_1^f}{\partial x_1^f} + T_1^f & x_1^f \frac{\partial T_1^f}{x_2^f} & x_1^f \frac{\partial T_1^f}{x_3^f} \\
x_1^f \frac{\partial T_2^f}{\partial x_1^f} + T_2^f & x_1^f \frac{\partial T_2^f}{x_2^f} & x_1^f \frac{\partial T_2^f}{x_3^f} \\
x_1^f \frac{\partial T_3^f}{\partial x_1^f} + T_3^f & x_1^f \frac{\partial T_3^f}{x_2^f} & x_1^f \frac{\partial T_3^f}{x_3^f} \\
x_2^f \frac{\partial T_2^f}{\partial x_1^f} & x_2^f \frac{\partial T_2^f}{x_2^f} + T_2^f & x_2^f \frac{\partial T_2^f}{x_3^f} \\
x_2^f \frac{\partial T_3^f}{\partial x_1^f} & x_2^f \frac{\partial T_3^f}{x_2^f} + T_3^f & x_2^f \frac{\partial T_3^f}{x_3^f} \\
x_3^f \frac{\partial T_3^f}{\partial x_1^f} & x_3^f \frac{\partial T_3^f}{x_2^f} & x_3 \frac{\partial T_3^f}{x_3^f} + T_3^f \\
\end{bmatrix}
%
\begin{bmatrix}
u_1^f \\ u_2^f \\ u_3^f 
\end{bmatrix} .
%
\end{eqnarray}
%
Furthermore, since $C_{ij}^f = C_{ji}^f$, the off diagonal terms (i.e., $C_{12}^f$, $C_{13}^f$, and $C_{23}^f$) can be expressed as averages:
\todo{why is C symmetric?}
%
\begin{eqnarray}
%
\begin{bmatrix}
x_1^f \frac{\partial T_1^f}{x_1^f} + T_1^f & x_1^f \frac{\partial T_1^f}{x_2^f} & x_1^f \frac{\partial T_1^f}{x_3^f} \\
%
\frac{1}{2} \left( x_1^f \frac{\partial T_2^f}{x_1^f} + x_2^f \frac{\partial T_1^f}{\partial x_1^f} + T_2^f \right) & \frac{1}{2} \left( x_1^f \frac{\partial T_2^f}{x_2^f} + x_2^f \frac{\partial T_1^f}{\partial x_2^f} + T_1^f \right) & \frac{1}{2} \left( x_1^f \frac{\partial T_2^f}{x_3^f} + x_2^f \frac{\partial T_1^f}{\partial x_3^f} \right) \\
%
\frac{1}{2} \left( x_1^f \frac{\partial T_3^f}{x_1^f} + x_3^f \frac{\partial T_1^f}{\partial x_1^f}  +T_3^f \right) & \frac{1}{2} \left(x_1^f \frac{\partial T_3^f}{x_2^f} + x_3^f \frac{\partial T_1^f}{x_2^f} \right) & \frac{1}{2} \left( x_1^f \frac{\partial T_3^f}{x_3^f} + x_3^f \frac{\partial T_1^f}{x_3^f} + T_1^f \right)\\
%
x_2^f \frac{\partial T_2^f}{x_1^f} & x_2^f \frac{\partial T_2^f}{x_2^f} + T_2^f & x_2^f \frac{\partial T_2^f}{x_3^f} \\
%
\frac{1}{2} \left( x_2^f \frac{\partial T_3^f}{x_1^f} + x_3^f \frac{\partial T_2^f}{x_1^f} \right)& \frac{1}{2} \left( x_2^f \frac{\partial T_3^f}{x_2^f} + x_3^f \frac{\partial T_2^f}{x_2^f} + T_3^f \right) & \frac{1}{2} \left(x_2^f \frac{\partial T_3^f}{x_3^f} + x_3^f \frac{\partial T_2^f}{x_3^f} + T_2^f \right)\\
%
x_3^f \frac{\partial T_3^f}{x_1^f} & x_3^f \frac{\partial T_3^f}{x_2^f} & x_3^f \frac{\partial T_3^f}{x_3^f} + T_3^f \\
\end{bmatrix}
%
\begin{bmatrix}
u_1^f \\ u_2^f \\ u_3^f 
\end{bmatrix}. \nonumber\\ 
%
\label{Cu}
\end{eqnarray}
%
Note that $\partial T_i^f/\partial x_j^f$ is the $i$, $j^{th}$ element of Eq.\ \eqref{K^e_aa_matrix} when $f=a$. Note that the matrix explicitly written out in Eq.\ \eqref{Cu} constitutes a single fiber node of the variable ttdSdy. The full ttdSdy variable, which is of size 6 $\times$ num\_dof, is constructed when summing over all boundary nodes (see summation in Eq.\ \eqref{DS_ij}).

\subsubsection{Directional Derivative of $1/V$}

In order to determine the directional derivative of $S_{ij}$ in Eq.\ \eqref{DS_ij}, we need to also determine $D \left(1/V\right) [\textbf{u}]$:
%

\begin{equation}
D \frac{1}{V}[\textbf{u}] = \frac{\partial (1/V)}{\partial x_i^f} u_i^f = -\frac{1}{V^2} \left( \frac{\partial V}{\partial x_i^f} u_i^f\right).
\label{D1/V}
\end{equation}
%
Equation \eqref{D1/V} can be further written in matrix form as:
%
\begin{eqnarray}
D\frac{1}{V}[\textbf{u}] = -\frac{1}{V^2}
\begin{bmatrix}
\frac{\partial (V)}{\partial x_1^f} & \frac{\partial (V)}{\partial x_2^f} & \frac{\partial(V)}{\partial x_3^f}
\end{bmatrix}
%
\begin{bmatrix}
u_1^f \\ u_2^f \\ u_3^f
\end{bmatrix}.
\label{D1/V_matrix}
\end{eqnarray}
%

\subsubsection{Combining it all together}

Substituting Eq.\ \eqref{D1/V} in Eq.\ \eqref{DS_ij} we obtain
%
\begin{eqnarray}
DS_{ij}[\textbf{u}] &=&  \frac{1}{V} \sum_{\text{bcl}} \left(x_i^f \frac{\partial T_j^f}{\partial x_k^f} u_k^f + \frac{\partial x_i^f}{\partial x_k^f}u_k^f T_j^f \right) -\frac{1}{V^2} \left( \frac{\partial V}{\partial x_k^f} u_k^f\right) V S_{ij} \nonumber\\
%
&=&  \frac{1}{V} \left[ \sum_{\text{bcl}} \left(x_i^f \frac{\partial T_j^f}{\partial x_k^f} + \frac{\partial x_i^f}{\partial x_k^f} T_j^f \right) -  \frac{\partial V}{\partial x_k^f} S_{ij} \right] u_k^f
\end{eqnarray}
%